\documentclass[11pt, a4paper]{article}

% --- PACKAGES ---
\usepackage[utf8]{inputenc} % Input encoding
\usepackage[T1]{fontenc}    % Font encoding
\usepackage{amsmath}        % Math environments
\usepackage{amssymb}        % Math symbols like \neg
\usepackage[english]{babel} % Language
\usepackage{csquotes}       % Recommended for biblatex with babel
\usepackage[margin=1in]{geometry} % Page margins
\usepackage{xcolor}         % Colors
\usepackage{indentfirst}    % Indent first paragraph after section title
\usepackage{booktabs}       % For professional quality tables
\usepackage{hyperref}       % Clickable links and references
\usepackage[
    backend=biber,
    style=numeric,
    sorting=nyt
]{biblatex}
\addbibresource{political_sat_paper.bib} % Specify the .bib file

% --- HYPERREF SETUP ---
% Setup for PDF metadata and link colors
\hypersetup{
    colorlinks=true,
    linkcolor=blue,           % Color for internal links (sections, figures)
    filecolor=magenta,        % Color for file links
    urlcolor=cyan,            % Color for URLs
    citecolor=red,            % Color for bibliography citations
    pdftitle={Formalizing Global Politics as a SAT Problem: Insights from Resolution, DP, and DPLL},
    pdfauthor={Razvan Goidaci},
    pdfsubject={SAT Solvers for Political Modeling, Logical Consistency and Scenario Analysis},
    pdfkeywords={SAT, DPLL, Resolution, DP, Politics, Modeling, Logic, Scenario Analysis, Consistency, Romania, Timisoara, GitHub},
}

% --- TITLE AND AUTHOR ---
\title{Formalizing Global Politics as a SAT Problem: Insights from Resolution, DP, and DPLL}
\author{
    Razvan Goidaci \\
    \texttt{razvan.goidaci05@e-uvt.ro} \\ \\
    Artificial Intelligence Specialization \\
    Facultatea de Matematică și Informatică \\
    Universitatea de Vest din Timișoara (UVT) \\
    Timișoara, Romania
}
\date{\today}

% --- DOCUMENT START ---
\begin{document}

\maketitle

% --- ABSTRACT ---
\begin{abstract}
This paper explores how Boolean Satisfiability (SAT) techniques can model global political problems. We show how to translate political statements and conditions into logical expressions in Conjunctive Normal Form (CNF). We then use our Python implementations of three classic SAT algorithms—Propositional Resolution \cite{Robinson1965}, the Davis-Putnam (DP) procedure \cite{DavisPutnam1960}, and the Davis–Putnam–Logemann–Loveland (DPLL) algorithm \cite{DPLL1962}—to check if these political models are logically consistent. More than just checking for contradictions, this work also looks into how changing the initial political assumptions affects the models and whether different political scenarios are logically possible. By showing how different starting conditions change the results, we point out important factors in these simplified political examples. This research links basic political modeling with computer-based logic, offering a clearer view of the difficulties and possibilities when using formal methods to understand the logical structure of political discussions and decisions.
\end{abstract}

% --- SECTIONS ---
\section{Introduction}

Global politics involves complex interactions, negotiations, and conflicts where decisions are often based on intricate sets of conditions and potential consequences. Understanding the logical consistency of political positions or the implications of certain actions can be profoundly challenging. This paper explores the enhanced application of tools from computer science, specifically Boolean Satisfiability (SAT) solvers, to analyze simplified models of political scenarios. The core idea is to represent political statements and rules using propositional logic, translate them into a standard format (Conjunctive Normal Form - CNF), and then use algorithms to check if there's a way for all these rules to be true simultaneously (i.e., if the model is satisfiable). We utilize three well-known algorithms for this task: Propositional Resolution \cite{Robinson1965}, the Davis-Putnam (DP) procedure \cite{DavisPutnam1960}, and the more modern Davis-Putnam-Logemann-Loveland (DPLL) algorithm \cite{DPLL1962}. We have implemented these algorithms in Python (source code available in Appendix \ref{app:sourcecode_access}). To illustrate this approach, we use examples drawn from international relations, such as imposing sanctions or forming alliances, modeling these situations with logical rules and converting them to CNF clauses to check their consistency. Input data for these case studies is also provided (see Appendix \ref{app:input_data}). A significant portion of this work is dedicated to exploring more involved, though still highly simplified, case studies, including one inspired by the Russia-Ukraine conflict. The objective extends beyond merely testing if a specific set of modeled assumptions leads to logical contradictions or allows for consistent scenarios. We aim to demonstrate the analytical power of SAT solvers in performing sensitivity analysis—examining how systematic variations in input clauses (political assumptions) alter the satisfiability of the model and highlight critical pivot points in potential outcomes. This "what-if" analysis is crucial for understanding the conditions under which different political outcomes become logically feasible or impossible within the defined model. This approach offers a novel lens to dissect the logical architecture of political discourse, potentially revealing non-obvious implications or hidden inconsistencies that might be overlooked by traditional qualitative analysis, thereby moving towards evaluating the potential of SAT solvers as a preliminary diagnostic tool for political analysts.
\subsection*{Reading Instructions}
Section \ref{sec:formal} explains the basic concepts of propositional logic and CNF used for modeling. Section \ref{sec:implementation} describes our Python implementations of the SAT solvers and how problems are represented and input. Section \ref{sec:casestudies} presents the enhanced case studies, showing the application of the solvers for consistency checking and scenario analysis. Section \ref{sec:relatedwork} discusses related areas of work, further situating our contribution. Section \ref{sec:conclusions} concludes with a summary of our findings and expanded future research directions. The Appendices provide links to the source code and case study input files.
\subsection*{Declaration of Originality}
The work presented in this paper is my own. The Python implementations of the Resolution, DP, and DPLL algorithms (referenced in Appendix \ref{app:sourcecode_access}) were developed from scratch based on standard algorithm descriptions \cite{Robinson1965, DavisPutnam1960, DPLL1962}. The methodology for applying these solvers to analyze simplified political models, including the development and extended analysis of the case studies presented, is my original contribution. Standard algorithms and concepts are referenced appropriately.

\section{Formal Description of Problem/Solution} \label{sec:formal}

To analyze a political scenario using SAT, we first need to translate it into the language of propositional logic.
\begin{itemize}
    \item \textbf{Propositions:} We identify key statements or conditions and represent them with propositional variables (usually single uppercase letters). For example, $S$ could represent "Country X imposes sanctions," or $A$ could represent "Treaty Y is signed." Each variable can be either True or False.
    \item \textbf{Logical Operators:} We connect these propositions using standard logical operators:
    \begin{itemize}
        \item $\neg$ (NOT): Negation (e.g., $\neg S$ means "Country X does \textit{not} impose sanctions").
        \item $\wedge$ (AND): Conjunction (e.g., $A \wedge B$ means "A is true AND B is true").
        \item $\vee$ (OR): Disjunction (e.g., $A \vee B$ means "A is true OR B is true OR both are true").
        \item $\Rightarrow$ (IMPLIES): Implication (e.g., $A \Rightarrow B$ means "If A is true, then B must also be true." This is logically equivalent to $\neg A \vee B$).
    \end{itemize}
    \item \textbf{Clauses and CNF:} Political rules and dependencies are often expressed as implications or other logical formulas. To use standard SAT solvers, these formulas must be converted into Conjunctive Normal Form (CNF). A CNF formula is an AND of clauses, where each clause is an OR of literals (a literal is a variable or its negation).
    \begin{itemize}
            \item Example: The statement "If a violation ($V$) occurs, then either sanctions ($S$) are imposed or aid ($R$) is reduced" can be written as $V \Rightarrow (S \vee R)$.
            \item This is equivalent to $\neg V \vee (S \vee R)$, which is already a single clause: $(\neg V \vee S \vee R)$.
    \end{itemize}
    \item \textbf{The SAT Problem:} Given a set of clauses in CNF representing the rules and facts of our political model, the SAT problem is to determine if there exists an assignment of True/False values to the variables such that \textit{all} clauses are simultaneously true.
    \begin{itemize}
        \item If such an assignment exists, the set of clauses (and the model it represents) is \textbf{Satisfiable (SAT)}.
        \item If no such assignment exists (meaning the rules lead to a logical contradiction), the set is \textbf{Unsatisfiable (UNSAT)}.
    \end{itemize}
\end{itemize}

\subsection*{Justification for SAT}
The choice of Boolean Satisfiability as the foundational framework for this exploratory analysis is deliberate. Its focus on determining the existence of a satisfying assignment for a set of logical constraints makes it particularly adept at rigorously testing the logical coherence of complex sets of political rules and conditions. Unlike some other formal methods, such as game theory which often presupposes actor rationality and utility functions \cite{BuenoDeMesquita2003}, SAT allows for an examination of the logical consequences of stated premises in a more direct fashion. This is especially useful in political contexts where rules, treaties, and public stances may have intricate logical interdependencies that are not immediately apparent. By focusing on satisfiability, we can identify potential contradictions or necessary consequences within a given political framework, offering a foundational layer of logical analysis before more complex factors like probabilities or actor intentions are introduced.
\subsection*{Algorithms Used}

\begin{itemize}
    \item \textbf{Resolution:} This method \cite{Robinson1965} repeatedly applies the resolution rule: From two clauses $(A \vee l)$ and $(B \vee \neg l)$, we can infer the new clause $(A \vee B)$. If we can eventually derive the empty clause ($\{\}$ or $\Box$, representing False), the original set was UNSAT. It's a complete method for proving unsatisfiability. Our implementation is referenced in Appendix \ref{app:resolution_code_link}.
    \item \textbf{Davis-Putnam (DP):} An older algorithm \cite{DavisPutnam1960} that uses rules like the one-literal rule (if a clause is just $\{l\}$, then $l$ must be true) and the pure literal rule (if a literal $l$ appears but $\neg l$ never does, assign $l$ to true) to simplify the problem, potentially combined with resolution. Our implementation is referenced in Appendix \ref{app:dp_code_link}.
    \item \textbf{Davis-Putnam-Logemann-Loveland (DPLL):} A more efficient, backtracking-based algorithm \cite{DPLL1962} widely used today. It uses unit propagation (similar to the one-literal rule) and then guesses a value for a variable (splitting), recursively trying to solve the simplified problem. If a guess leads to a contradiction, it backtracks and tries the opposite value. Our implementation is referenced in Appendix \ref{app:dpll_code_link}.
\end{itemize}

\subsection*{Correctness and Complexity}

\begin{itemize}
    \item \textbf{Correctness:} The conversion to CNF maintains logical equivalence. The Resolution, DP, and DPLL algorithms are sound (they don't claim UNSAT for a SAT set) and complete (they will eventually determine satisfiability or unsatisfiability). The \textit{meaningfulness} of the result, however, depends entirely on how accurately and comprehensively the initial political scenario was modeled.
    \item \textbf{Complexity:} SAT is generally a hard problem (NP-complete). In the worst case, these algorithms can take exponential time relative to the number of variables. However, DPLL and modern solvers often perform surprisingly well on many practical problems, especially those with underlying structure. See \cite{Cormen1996} for general algorithm complexity discussions.
\end{itemize}

\subsection*{Example (Sanction Problem Revisited)}
Given the political statements:
\begin{enumerate}
    \item If a violation ($V$) occurs, then either sanctions ($S$) are imposed or aid ($R$) is reduced ($V \Rightarrow (S \vee R)$).
    \item A violation ($V$) has occurred ($V$).
    \item Sanctions ($S$) have \textit{not} been imposed ($\neg S$).
\end{enumerate}

Convert to CNF clauses:
\begin{enumerate}
    \item $\neg V \vee S \vee R$
    \item $V$
    \item $\neg S$
\end{enumerate}

Clause Set Representation (as used in Python input, see Appendix \ref{app:case1_input_link} for file): \texttt{[['¬V', 'S', 'R'], ['V'], ['¬S']]}. We can use a SAT solver to see if these conditions are consistent, and further, what implications arise.

\section{Modelling the Problem / Implementing the Solution} \label{sec:implementation}
% This section remains unchanged as requested.
\subsection*{Representing the Problem}
In our Python implementations (referenced in Appendix \ref{app:sourcecode_access}), a set of clauses is represented as a list of lists. Each inner list contains strings representing the literals in that clause. For example: \texttt{[['¬A', 'B'], ['A', 'C']]} represents $(\neg A \vee B) \wedge (A \vee C)$.
\subsection*{Implementation Details}
We implemented the three solvers (Resolution, DP, DPLL) in Python (referenced in Appendix \ref{app:sourcecode_access}). The algorithms are based on the original works by Robinson \cite{Robinson1965}, Davis and Putnam \cite{DavisPutnam1960}, and Davis, Logemann, and Loveland \cite{DPLL1962}.
\begin{itemize}
    \item The code includes functions to check literal format, find complements ($\neg A$ vs $A$), perform resolution (`union`), and implement the specific rules for each algorithm (one-literal, pure literal, splitting).
    \item The scripts take user input to define the clause set and then run the respective algorithm, printing intermediate steps (if applicable) and the final result. For the case studies, these inputs were manually curated to represent the scenarios, and are available as text files (see Appendix \ref{app:input_data}).
\end{itemize}
\subsection*{Programming Manual}
\begin{itemize}
    \item \textbf{Files:} The Python scripts \texttt{Propositional Resolution.py}, \texttt{DP.py}, and \texttt{DPLL.py} are available online (see Appendix \ref{app:sourcecode_access}).
    \item \textbf{Language:} Python 3.
    \item \textbf{Dependencies:} `re` and `copy` standard libraries.
    \item \textbf{Core Logic:} Each file contains functions implementing the rules of the corresponding SAT algorithm, operating on the list-of-lists clause set representation. The \texttt{create\_clause\_set} function handles user input.
\end{itemize}
\subsection*{User Manual}
To use one of the solvers:
\begin{enumerate}
    \item Download the desired Python script from the GitHub repository (see Appendix \ref{app:sourcecode_access}).
    \item Run the script from your terminal (e.g., `python DPLL.py`).
    \item The script will first ask: \texttt{How many clauses are in the clause set?:}. Enter the total number of clauses.
    \item The script will then prompt you to input each clause, one by one (e.g., \texttt{Input clause number 1:}).
    \item Type the literals for that clause separated by spaces (e.g., \texttt{¬P ¬S} or \texttt{A B ¬C}). Ensure literals use the format `A`, `B`, ... for positive literals and `¬A`, `¬B`, ... for negative literals. Press Enter after each clause.
    \item After you input all clauses, the script will execute the SAT algorithm.
    \item It may print intermediate steps showing how the clause set is modified.
    \item The final output will state whether the input clause set is \textbf{Satisfiable (SAT)} or \textbf{Unsatisfiable (UNSAT)}.
\end{enumerate}
\textbf{Note on Batch Input for Larger Clause Sets:}
For larger sets of clauses, to avoid manually entering each clause and the total count through separate prompts, you can prepare your input as a single block of text. This entire block should be pasted directly into the \textit{first input prompt} (when the script asks: \texttt{How many clauses are in the clause set?:}). Examples of this format are provided in the input files referenced in Appendix \ref{app:input_data}. The format for this text block should be:
\begin{itemize}
    \item Line 1: The total number of clauses in the set.
    \item Subsequent Lines: Each line should represent a single clause, with literals separated by spaces. There should be as many clause lines as the number specified on the first line.
\end{itemize}
For example, to input a set with 3 clauses:
\begin{verbatim}
3
A B
A B C
¬A ¬B
\end{verbatim}
After pasting this entire block into the first input prompt and pressing Enter, the program should parse this input and proceed correctly. This method assumes the Python script's input handling is designed to support multi-line paste parsing at the initial prompt.

\section{Case Studies / Experiments: Enhanced Analysis} \label{sec:casestudies}

Here we apply the implemented solvers to analyze the consistency and explore scenario variations within simplified political models. The goal is not only to determine satisfiability but also to understand the implications of different assumptions. The input data for these base scenarios can be found in Appendix \ref{app:input_data}.

\subsection*{Quick Reference for Case Study Variables}
To aid comprehension, the variables used across the case studies are summarized in the table below.

\begin{table}[h!]
\centering
\caption{Case Study Variable Definitions}
\label{tab:variable_definitions}
\begin{tabular}{@{}ll@{}}
\toprule
\textbf{Variable} & \textbf{Description} \\ \midrule
\multicolumn{2}{l}{\textbf{Case 1: Sanctions and Aid}} \\
V & A violation occurs \\
S & Sanctions are imposed \\
R & Aid is reduced \\ \midrule
\multicolumn{2}{l}{\textbf{Case 2: NATO Membership}} \\
N & Country A joins NATO \\
M & Country B increases military spending \\
T & Country C withdraws from a peace treaty \\ \midrule
\multicolumn{2}{l}{\textbf{Case 3: Conflict Resolution}} \\
I ($I$) & Russia invades Ukraine \\
R ($R_U$) & Ukraine resists \\
S ($S_{NATO}$) & NATO sends military support to Ukraine \\
P ($P$) & Peace talks begin \\
W ($W_U$) & Ukraine wins the war (model-defined) \\
C ($C_{Crimea}$) & Crimea is returned to Ukraine \\
B ($B_{Blockade}$) & Russia sets up a blockade \\ \bottomrule
\end{tabular}
\end{table}

\subsection*{Case 1: Sanctions and Aid – Deeper Implications}
\begin{itemize}
    \item \textbf{Scenario:} Based on the example in Section \ref{sec:formal}. (Input file: Appendix \ref{app:case1_input_link})
    \begin{itemize}
        \item If a violation ($V$) occurs, then sanctions ($S$) are imposed or aid ($R$) is reduced ($V \Rightarrow (S \vee R)$).
        \item Violation ($V$) occurred ($V$).
        \item Sanctions ($S$) were not imposed ($\neg S$).
    \end{itemize}
    \item \textbf{Base Clause Set (from \texttt{case1\_sanctions\_aid\_input.txt}):} $C_1 = \{\{\neg V, S, R\}, \{V\}, \{\neg S\}\}$.
    \item \textbf{Base Analysis:} Using unit propagation, the clauses force the assignment $V$=True and $S$=False. This simplifies the first clause to $\{\text{False} \vee \text{False} \vee R\}$, leaving the requirement that $\{R\}$ must be true.
    \item \textbf{Base Result:} The set is \textbf{Satisfiable (SAT)} with the model $V$=True, $S$=False, $R$=True.
    \item \textbf{Base Interpretation:} The conditions are logically consistent. A key insight is that for all rules to hold, Aid ($R$) \textit{must} be reduced ($R$=True). This is not just a possibility, but a logical necessity of the model.
    \item \textbf{Scenario Variation 1.1: What if aid is also not reduced?}
        \begin{itemize}
            \item \textbf{Change:} Add the premise that Aid ($R$) is \textit{not} reduced ($\neg R$).
            \item \textbf{New Clause Set:} $C_{1.1} = C_1 \cup \{\{\neg R\}\} = \{\{\neg V, S, R\}, \{V\}, \{\neg S\}, \{\neg R\}\}$.
            \item \textbf{Analysis:} The base analysis derived that $R$ must be true. The new premise $\{\neg R\}$ directly contradicts this necessary conclusion, leading to the derivation of the empty clause.
            \item \textbf{Result:} $C_{1.1}$ is \textbf{Unsatisfiable (UNSAT)}.
            \item \textbf{Interpretation:} Within this model, it is logically impossible for a violation to occur, sanctions to be withheld, AND aid to be maintained. At least one of the initial conditions must be false, or the governing rule ($V \Rightarrow (S \vee R)$) must be broken. This highlights the direct trade-off embedded in the initial rule.
        \end{itemize}
\end{itemize}

\subsection*{Case 2: NATO Membership and Regional Stability – Exploring Alternatives}
\begin{itemize}
    \item \textbf{Scenario:} A simplified model of tensions related to alliance expansion. (Input file for base analysis: Appendix \ref{app:case2_input_link})
    \begin{itemize}
        \item If Country A joins NATO ($N$), then Country B increases military spending ($M$) ($N \Rightarrow M$).
        \item If Country A joins NATO ($N$) AND Country B increases military spending ($M$), then Country C withdraws from a peace treaty ($T$) ($(N \wedge M) \Rightarrow T$).
        \item Country C has \textit{not} withdrawn from the treaty ($\neg T$).
    \end{itemize}
    \item \textbf{Goal:} Is it consistent for Country A to have joined NATO ($N$=True) under these conditions?
    \item \textbf{Base CNF Clauses (from \texttt{case2\_nato\_stability\_input.txt}, which includes $N$ as a premise):} $C_2 = \{\{\neg N, M\}, \{\neg N, \neg M, T\}, \{\neg T\}, \{N\}\}$.
    \item \textbf{Base Analysis (with $N$=True as a premise):}
        The premise $N$=True initiates a chain of deductions:
        \begin{itemize}
            \item $\{N\}$ and $\{\neg N, M\}$ resolve to $\{M\}$, forcing $M$ to be true.
            \item With $N$=True and the derived $M$=True, rule 2 implies $T$ must be true. Formally, $\{N\}$ and $\{\neg N, \neg M, T\}$ gives $\{\neg M, T\}$, which when resolved with $\{M\}$ gives $\{T\}$.
            \item This derived conclusion, $\{T\}$, directly contradicts the initial premise $\{\neg T\}$, leading to the empty clause.
        \end{itemize}
    \item \textbf{Base Result:} The input set is \textbf{Unsatisfiable (UNSAT)}.
    \item \textbf{Base Interpretation:} Within this model, it's logically impossible for Country A to join NATO given the other rules and the fact that Country C did not withdraw from the treaty. The contradiction arises from the implication chain: $N$ forces $M$, which together force $T$, directly conflicting with the given fact $\neg T$.
\end{itemize}

\subsection*{Case 3: Complex Conflict Resolution — Russia–Ukraine War (Highly Simplified Model)}
(Input file for base scenario: Appendix \ref{app:case3_input_link})
\begin{itemize}
    \item \textbf{Disclaimer:} \textit{This case study uses a highly simplified model of an extremely complex real-world conflict. It is intended ONLY to demonstrate the application of SAT solving to a set of logical rules for consistency checking and preliminary scenario exploration, NOT to provide realistic political analysis, predict outcomes, or reflect the full spectrum of views on the conflict. Variables are defined in Table \ref{tab:variable_definitions}.}
    \item \textbf{Statements (Simplified Rules - Base Model $R_0$):}
    \begin{enumerate}
        \item If Russia invades ($I$) and Ukraine doesn't resist ($\neg R_U$), Ukraine doesn't win ($\neg W_U$) ($(I \wedge \neg R_U) \Rightarrow \neg W_U$).
        \item If Ukraine resists ($R_U$) and receives NATO help ($S_{NATO}$), Ukraine can win ($W_U$) ($(R_U \wedge S_{NATO}) \Rightarrow W_U$).
        \item If Ukraine wins ($W_U$), Crimea is returned ($C_{Crimea}$) ($W_U \Rightarrow C_{Crimea}$).
        \item If peace talks begin ($P$), there is no NATO military support ($\neg S_{NATO}$) ($P \Rightarrow \neg S_{NATO}$).
        \item If there is a blockade ($B_{Blockade}$) and Ukraine resists ($R_U$), peace talks begin ($P$) ($(B_{Blockade} \wedge R_U) \Rightarrow P$).
        \item Fact: Peace talks have not started ($\neg P$).
        \item Fact: Russia has invaded ($I$).
    \end{enumerate}
    \item \textbf{Base Clause Set:} \texttt{[['¬I', 'R', '¬W'], ['¬R', '¬S', 'W'], ['¬W', 'C'], ['¬P', '¬S'], ['¬B', '¬R', 'P'], ['¬P'], ['I']]}.
    \item \textbf{Base Analysis:}
        \begin{itemize}
            \item The initial set of clauses is \textbf{Satisfiable (SAT)}.
            \item For example, one possible consistent scenario (a satisfying assignment) is: $I$=T, $R$=T, $S$=T, $W$=T, $C$=T, $P$=F, $B$=F.
            \item \textbf{Interpretation:} The modeled rules are not inherently contradictory. They allow for a logically consistent scenario where an invasion occurs, Ukraine resists and wins with NATO support, Crimea is returned, and peace talks have not started. This does \textbf{not} imply this outcome is probable, only that it is logically possible within the simplified framework.
        \end{itemize}
    \item \textbf{Scenario Variation 3.1: Impact of Peace Talks Starting ($P$=True)}
        \begin{itemize}
            \item \textbf{Change:} Replace premise $\neg P$ with $P$.
            \item \textbf{Analysis:} The rule $P \Rightarrow \neg S_{NATO}$ (Clause 4: $\neg P \vee \neg S$) combined with the new premise $P$ immediately forces the conclusion that NATO support is withdrawn ($\neg S_{NATO}$). Now, for Ukraine to win ($W_U$) through the mechanism in Rule 2, it would require $S_{NATO}$ to be true, creating a tension.
            \item \textbf{Result:} The set is still \textbf{Satisfiable (SAT)}. A satisfying assignment could be: $P$=T, $S$=F, $W$=F, $C$=F, $R$=T, $I$=T, $B$=F.
            \item \textbf{Interpretation:} If peace talks start, the model forces NATO support to cease. Under this condition, the path to a Ukrainian victory \textit{as defined by Rule 2} is blocked. The model remains consistent, but points to an outcome where Ukraine does not win.
        \end{itemize}
    \item \textbf{Scenario Variation 3.2: Impact of a Blockade ($B$=True) + Resistance ($R$=True)}
        \begin{itemize}
            \item \textbf{Change:} To the base model, add the premises $B$=True and $R$=True.
            \item \textbf{Analysis:} Rule 5 states $(B_{Blockade} \wedge R_U) \Rightarrow P$. With the new premises, this rule forces the conclusion that peace talks must begin ($P$=True). However, the base model contains the explicit premise that peace talks have \textit{not} started ($\neg P$). The model thus requires both $P$ and $\neg P$ to be true, which is a contradiction.
            \item \textbf{Result:} The resulting set is \textbf{Unsatisfiable (UNSAT)}.
            \item \textbf{Interpretation:} Given the initial rules and the fact that "Peace talks have not started," it's logically inconsistent for a blockade and Ukrainian resistance to occur simultaneously. This highlights a fundamental tension in the model.
        \end{itemize}
\end{itemize}

\subsection*{Summary of Case Study Outcomes}
The results of the scenario analyses are synthesized in the table below.

\begin{table}[h!]
\centering
\caption{Case Study Results Summary}
\label{tab:results_summary}
\begin{tabular}{@{}llll@{}}
\toprule
\textbf{Case Study} & \textbf{Scenario} & \textbf{Key Assumption/Change} & \textbf{Result} \\ \midrule
\textbf{Case 1} & Base & Violation occurs, no sanctions & SAT \\
(Sanctions) & Variation 1.1 & + Aid is not reduced & UNSAT \\ \midrule
\textbf{Case 2} & Base & Country A joins NATO & UNSAT \\
(NATO) & Variation 2.1 & + Treaty withdrawal is OK & SAT \\ \midrule
\textbf{Case 3} & Base & Invasion occurs, no peace talks & SAT \\
(Conflict) & Variation 3.1 & + Peace talks begin ($P$) & SAT \\
& Variation 3.2 & + Blockade ($B$) and Resistance ($R$) & UNSAT \\ \bottomrule
\end{tabular}
\end{table}

\section{Related Work} \label{sec:relatedwork}

Boolean Satisfiability is a cornerstone of computer science with applications in hardware verification, AI planning, and bioinformatics. Modern solvers like Z3 or Glucose use heuristics far beyond our basic implementations, handling millions of variables. While formal methods are used in legal reasoning and social choice theory \cite{Gehlbach2013}, their direct application to dynamic political conflicts using basic SAT is less common.

Game theory is more frequently employed for strategic analysis in political science, focusing on actor rationality, payoffs, and equilibria, as exemplified in works on political survival and strategy \cite{BuenoDeMesquita2003}. Agent-based modeling is another common computational approach. Our work differs by focusing on the logical consistency of stated rules rather than actor rationality. The unique niche of this SAT-based approach is its capacity to rigorously check the logical coherence of propositions that might represent policy statements or treaty obligations, without needing to model actor utilities or probabilities.

\textbf{Advantages:} Provides a clear, formal framework; allows for rigorous consistency checking; can identify non-obvious contradictions; enables systematic "what-if" exploration.

\textbf{Limitations:} Difficulty in accurately capturing the nuances of real-world politics; the manual translation to CNF can be subjective; ignores crucial factors like probabilities, beliefs, and power dynamics. The model is only as good as its formalization.

\section{Conclusions and Enhanced Future Work} \label{sec:conclusions}

This paper demonstrated how to model simplified political scenarios as SAT problems, analyzing consistency and exploring outcomes through scenario variation with implementations of Resolution, DP, and DPLL. The case studies showed how to translate political rules into CNF and use SAT solvers to determine if assumptions are consistent (SAT) or contradictory (UNSAT), and how changes in premises affect these conclusions. The primary challenge remains the modeling process itself: translating the multifaceted complexities of real-world politics into precise Boolean logic is difficult and requires significant simplification. The validity of any analysis depends critically on the fidelity of the underlying model. The analysis revealed how simple political models can have non-obvious logical consequences and how sensitive outcomes are to initial assumptions.

\subsection*{Enhanced Future Work}
The exploratory nature of this work opens several avenues for more developed research:
\begin{itemize}
    \item \textbf{Richer Modeling Logics:} Move beyond simple true/false statements. For example, modal logics could distinguish between what is \textit{necessarily} true versus what is merely \textit{possibly} true in a political arrangement. Probabilistic logics could represent uncertainty; for instance, a framework like Markov Logic Networks \cite{Richardson2006} could assign probabilities to rules, capturing the idea that a political 'rule' holds true with a certain likelihood.

    \medskip
    \item \textbf{Automation and NLP Integration:} Develop tools to assist in semi-automated translation from structured text (e.g., policy documents) into logical formulas. This remains a challenging NLP task but is crucial for broader applicability.

    \medskip
    \item \textbf{Interface and Visualization:} Create a graphical user interface (GUI) for easier input of rules, visualization of the logical model (e.g., implication graphs), and interactive exploration of "what-if" scenarios.

    \medskip
    \item \textbf{Advanced Solvers and Scalability:} Integrate state-of-the-art SAT/SMT solvers (e.g., Z3) to handle larger, more realistic models.

    \medskip
    \item \textbf{Optimization and Explanation (MaxSAT \& MUS):} Use MaxSAT to find assignments that satisfy the \textit{maximum number} of rules, useful for situations with contradictory goals. Employ techniques for finding Minimal Unsatisfiable Subsets (MUS) to pinpoint the core contradictions when a model is UNSAT.
\end{itemize}

% --- BIBLIOGRAPHY ---
\nocite{*}
\printbibliography

% --- APPENDICES ---
\appendix
\section{Appendix: Source Code Access} \label{app:sourcecode_access}

The complete project, including the LaTeX source for this paper, the BibTeX bibliography file, and the Python implementations of the SAT solvers, is available on GitHub at:
\url{https://github.com/razvangoidaci/Politacal-SAT-Paper}

The Python scripts for the implemented solvers can be accessed directly via the following links:

\subsection{Propositional Resolution Solver (\texttt{Propositional Resolution.py})} \label{app:resolution_code_link}
The source code for the Propositional Resolution solver is available at:
\url{https://github.com/razvangoidaci/Politacal-SAT-Paper/blob/main/Propositional%20Resolution.py}

\subsection{DP Solver (\texttt{DP.py})} \label{app:dp_code_link}
The source code for the Davis-Putnam (DP) solver is available at:
\url{https://github.com/razvangoidaci/Politacal-SAT-Paper/blob/main/DP.py}

\subsection{DPLL Solver (\texttt{DPLL.py})} \label{app:dpll_code_link}
The source code for the Davis-Putnam-Logemann-Loveland (DPLL) solver is available at:
\url{https://github.com/razvangoidaci/Politacal-SAT-Paper/blob/main/DPLL.py}

\section{Appendix: Case Study Input Data Files} \label{app:input_data}
The input files for the base scenarios of the case studies, formatted for use with the Python solvers (particularly with the batch input method described in the User Manual), are available on GitHub. These files use single uppercase letters for propositional variables.

\subsection{Case 1: Sanctions and Aid - Input File} \label{app:case1_input_link}
Input data for Case Study 1 (\texttt{case1\_sanctions\_aid\_input.txt}):
\url{https://github.com/razvangoidaci/Politacal-SAT-Paper/blob/main/case1_sanctions_aid_input.txt}

\subsection{Case 2: NATO Membership and Regional Stability - Input File} \label{app:case2_input_link}
Input data for Case Study 2 (\texttt{case2\_nato\_stability\_input.txt}):
\url{https://github.com/razvangoidaci/Politacal-SAT-Paper/blob/main/case2_nato_stabilty_input.txt}

\subsection{Case 3: Complex Conflict Resolution - Input File} \label{app:case3_input_link}
Input data for Case Study 3 (\texttt{case3\_conflict\_resolution\_input.txt}):
\url{https://github.com/razvangoidaci/Politacal-SAT-Paper/blob/main/case3_conflict_resolution_input.txt}


% --- DOCUMENT END ---
\end{document}
